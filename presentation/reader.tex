\section{Reader}
\subsection{Motivation}
\begin{frame}[fragile]{Reader Monad}{Motivation}
  \begin{columns}[t]
    \begin{column}{.4\textwidth}
      {\tiny\inputminted[escapeinside=\\`\\`]{haskell}{reader-motivation.hs}}
    \end{column}
    \begin{column}{.4\textwidth}
      \begin{itemize}
        \item Notice how we must pass this \texttt{\mylib{wc}} around
        \item We must pass it through functions that may never use it
      \end{itemize}
    \end{column}
  \end{columns}
\end{frame}
\subsection{Definition}
\begin{frame}[fragile]{Reader Monad}{Type}
  \inputminted[escapeinside=||,fontsize=\tiny]{haskell}{reader-types.hs}
  {
    \vspace*{-4cm}
    \hspace{7cm}
    \begin{minipage}{.35\textwidth}
      \begin{tcolorbox}[colframe=gray,colback=white,boxrule=1pt,arc=3.4pt,boxsep=0mm]
        \begin{minted}{csharp}
          class Reader<TC, TA> {
            Reader(T runReader) {
              this.runReader = runReader;
            }
          }
        \end{minted}
      \end{tcolorbox}
    \end{minipage}%
  }
  \vspace*{2.5cm}
  \begin{block}{\textbf{Exercise: }Reader Type Definition}
    \begin{align*}
      \mathsf{newtype}\ \ReaderH{a} &=
        \mathsf{Reader} \left\{ \fn{runReader} ::\ \uncover<2->{\mylib{c}\ \rightarrow\ \mylibo{a}} \right\}\\
    \end{align*}
  \end{block}
\end{frame}
\begin{frame}[fragile]{Reader Monad}{Type}
  Recall: $\mathsf{newtype}\ \Reader{a} = \mathsf{Reader} \left\{ \fn{runReader} ::\ c\ \rightarrow\ a \right\}$
  \begin{block}{\textbf{Exercise: }Reader Type Definition}
    \begin{align*}
      \return &:: a\ \rightarrow\ \Reader{a}\\
      \return\ a\ &= \uncover<2->{\mathsf{Reader}\ (\lambda\,\underline{\phantom{c}}\ \rightarrow\ a)}\\[1.5em]
      (\bind) &:: \Reader{a}\ \rightarrow\ (a\ \rightarrow\ \Reader{b})\ \rightarrow\ \Reader{b}\\
      x\ \bind\ fn &=\uncover<3->{\mathsf{Reader}\ (\lambda\,c\ \rightarrow
      \fn{runReader}\ (fn\ (\fn{runReader}\ x\ c))\ c)}\\
    \end{align*}
  \end{block}
\end{frame}
\begin{frame}[fragile]{Reader Monad}{Functions}
    \begin{block}{\textbf{Homework: }Reader Functions}
      \begin{align*}
        \fn{ask} &:: \Reader{a}\ \rightarrow\ a\\
        \fn{asks} &:: \Reader{a}\ \rightarrow\ (a\ \rightarrow\ \Reader{b})\ \rightarrow\ \Reader{b}\\
        \fn{local} &:: (c\ \rightarrow\ c\,')\ \rightarrow\ \mathbf{Reader}\,c\,'\,a\ \rightarrow\ \Reader{a}\\
      \end{align*}
    \end{block}
\end{frame}
\subsection{Example}
\begin{frame}
  \centering
  \begin{tcolorbox}[enhanced, size=minimal,auto outer arc,
    width=2.8cm,octogon arc, colback=red,colframe=white,colupper=white, fontupper=\fontsize{7mm}{7mm}\selectfont\bfseries\sffamily, halign=center,valign=center,
    square,arc is angular, borderline={1mm}{-3mm}{red} ]
    CODE
  \end{tcolorbox}
\end{frame}
