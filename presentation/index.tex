\documentclass{beamer}
\PassOptionsToPackage{usenames,dvipsnames}{xcolor}
\usepackage{xcolor}
\usetheme{Boadilla}
\usepackage[cache=false]{minted}
\usepackage{multicol}
\usepackage{latexsym}
\usepackage{amsmath}
\usepackage{amssymb}
\usepackage{graphicx}
\usepackage[T1]{fontenc} %
\usepackage[skins,most]{tcolorbox}
\usepackage{mathtools}
\usepackage{tikz}
\usepackage{addfont}
\addfont{OT1}{d7seg}{\dviiseg}
\usepackage{fontspec}
\usepackage{changepage}
\let\origcheckmark\checkmark % save the macro 
\let\checkmark\relax         % reset it, so it can be defined using `\newcommand` can be used without error
\usepackage{dingbat}     % load package
\let\altcheckmark\checkmark  % save that macro definition under a different name if required
\let\checkmark\origcheckmark 
\usetikzlibrary{decorations.text, shadows}
% \newfontfamily\looney[]{Folks}

\definecolor{darkblueOuter}{RGB}{1,11,23}
\definecolor{darkblueInner}{RGB}{1,18,37}

\graphicspath{{./images/}}
\newcommand\id{\mathsf{id}}
\newcommand\return{\mathsf{return}}
\newcommand\bind{\mathsf{>\!\!>\!\!=}}
\newcommand\app{\mathsf{<\!\!*\!\!>}}
\newcommand\unit{()}
\newcommand\pure{\mathsf{pure}}
\newcommand\fmap{\mathsf{fmap}}
\newcommand\join{\mathsf{join}}
\newcommand\oo{\ensuremath{\,\circ\,}}
\newcommand\om{\ensuremath{\,\odot\,}}
\newcommand{\M}[1]{\ensuremath{\mathsf{M}\,#1}}
\newcommand{\F}[1]{\ensuremath{\mathsf{F}\,#1}}
\newcommand{\A}[1]{\ensuremath{\mathsf{A}\,#1}}

% \tcbset{every box/.style={highlight math style={boxrule=0pt,arc=3pt, left=0pt,right=0pt,top=0pt,bottom=0pt,boxsep=2pt}} }

\definecolor{BlueGreen}{rgb}{0.0, 0.87, 0.87}
\definecolor{Amber}{rgb}{1.0, 0.49, 0.0}
\newtcbox{\mylib}{enhanced,nobeforeafter,tcbox raise base,boxrule=0.4pt,top=0.5pt,bottom=0.5pt,
  right=0mm,left=0mm,arc=1pt,boxsep=1pt,before upper={\vphantom{dlg}},
  colframe=BlueGreen!50!white,coltext=BlueGreen!25!black,colback=BlueGreen!50,
  overlay={\begin{tcbclipinterior}(frame.north west);\end{tcbclipinterior}}}

\robustify{\mylib}

\pdfstringdefDisableCommands{%
  \def\mylib#1{'#1'}%
}

\newtcbox{\mylibo}{enhanced,nobeforeafter,tcbox raise base,boxrule=0.4pt,top=0.5pt,bottom=0.5pt,
  right=0mm,left=0mm,arc=1pt,boxsep=1pt,before upper={\vphantom{dlg}},
  colframe=Amber!50!white,coltext=Amber!25!black,colback=Amber!50,
  overlay={\begin{tcbclipinterior}(frame.north west);\end{tcbclipinterior}}}

\robustify{\mylibo}

\pdfstringdefDisableCommands{%
  \def\mylibo#1{'#1'}%
}

\immediate\write18{sh ./scripts/reader.sh > reader-motivation.hs}
\immediate\write18{sh ./scripts/reader-types.sh > reader-types.hs}
\immediate\write18{sh ./scripts/state.sh > state-motivation.hs}
\immediate\write18{sh ./scripts/state-types.sh > state-types.hs}
\immediate\write18{sh ./scripts/writer.sh > writer-motivation.hs}
\immediate\write18{sh ./scripts/writer-types.sh > writer-types.hs}

\setminted[csharp]{baselinestretch=.8,fontsize=\tiny,autogobble}
\makeatletter
\newenvironment{code}
 {\RecustomVerbatimEnvironment{Verbatim}{BVerbatim}{}%
  \def\FV@BProcessLine##1{%
    \hbox{%
      \hbox to\z@{\hss\theFancyVerbLine\kern\FV@NumberSep}%
      \FancyVerbFormatLine{##1}%
    }%
  }%
  \VerbatimEnvironment
  \setbox\z@=\hbox\bgroup
  \begin{minted}{csharp}}
 {\end{minted}\egroup
  \leavevmode\vbox{\box\z@}}
\makeatother

\begin{document}
\titlegraphic{\vspace{-3em}\includegraphics[width=.3\textwidth]{hunted-wumpus.jpg}}
\title{Hunt the Trevor}
\author{Elliot Greenwood}
\date{}
  \begin{frame}[fragile]
    \titlepage%
  \end{frame}
  \begin{frame}{Agenda}
    \begin{columns}[t]
      \begin{column}{.5\textwidth}
        \tableofcontents[sections={1-3}]
      \end{column}
      \begin{column}{.4\textwidth}
        \tableofcontents[sections={4-6}]
      \end{column}
    \end{columns}
  \end{frame}
  \section{What is Hunt the Wumpus?}
  \begin{frame}{What is Hunt the Wumpus?}{Why are we Hunting Poor Trevor?}
    You, the brave adventurer, have gotten lost and stumbled into a dark cavern.
    You can barely see your hand in front of your face.
    You heard rumours that Trevor, the Wumpus, lived in these here parts.\\[1em]
    Ahead of you are:
    \begin{itemize}[<+-| alert@+>]
      \item 20 Caves, each connected to 3 other caves
      \item 5 Arrows (a.k.a. 5 attempts to kill Trevor)
      \item 2 crazy bats (that like to transport you to other caves)
      \item 2 endless pits (that almost certainly mean death)
      \item Some great fun from \dviiseg{1973}
    \end{itemize}
    \pause{On your turn you can Move to an adjoining cave, via a dark tunnel,
      or shoot one of your crooked arrows in the hope of hitting the Wumpus.}\\
    \pause{Fortunately, the bats make a lot of noise,
      the pits cause an awful draft, \& Trevor is quite smelly!}
  \end{frame}
  \section{What is a Monad?}
  \subsection{Intuition}
  \begin{frame}[fragile]{What is a Monad?}{Intuition}
    \begin{columns}[t]
      \begin{column}{.9\textwidth}
        \begin{itemize}
          \item The functional programming style forces you to expose your inputs, outputs and intentions
          \item Monads help cover them up again
          \item They can then allow you to seamlessly introduce abstractions,
            \textit{only where they are needed}, letting you focus on the business logic
          \item They provide convenient frameworks for effects found in imperative languages\,\footnotemark
            (e.g. raising exceptions,
            null checking,
            random number generators,
            and, \textit{cough}, I/O)
        \end{itemize}
      \end{column}
      \begin{column}{.1\textwidth}
      \end{column}
    \end{columns}
    \footnotetext{\scriptsize Wadler, P. (1995), Monads for Functional Programming, in 'Advanced Functional Programming' , Springer, London , pp. 24--52 .}
  \end{frame}
  \subsection{Theory}
  \begin{frame}{What is a Monad?}{Theory}
    \begin{block}{Monad Definition}
      \begin{align*}
        \return &:: \mathbf{Monad}\ \mathsf{m}\ \Rightarrow\ a\ \rightarrow\ \M{a}\\
        (\bind) &:: \mathbf{Monad}\ \mathsf{m}\ \Rightarrow\ \M{a}\ \rightarrow\ (a\ \rightarrow\ \M{b})\ \rightarrow\ \M{b}\\
      \end{align*}
    \end{block}
    \begin{itemize}
      \item $\return$ constructs the monad from a value\\
      \item $\bind$ allows composition of the monad\\
      \item \textit{Note: You might see other people say you need $\join$ to define the monad,
        providing either $\bind$ or $\join$ is ``rigorously'' equivalent.}
    \end{itemize}
  \end{frame}
  \section{Reader}
\subsection{Motivation}
\begin{frame}[fragile]{Reader Monad}{Motivation}
  \begin{columns}[t]
    \begin{column}{.4\textwidth}
      {\tiny\inputminted[escapeinside=\\`\\`]{haskell}{reader-motivation.hs}}
    \end{column}
    \begin{column}{.4\textwidth}
      \begin{itemize}
        \item Notice how we must pass this \texttt{\mylib{wc}} around
        \item We must pass it through functions that may never use it
      \end{itemize}
    \end{column}
  \end{columns}
\end{frame}
\subsection{Definition}
\begin{frame}[fragile]{Reader Monad}{Type}
  \inputminted[escapeinside=||,fontsize=\tiny]{haskell}{reader-types.hs}
  {
    \vspace*{-4cm}
    \hspace{7cm}
    \begin{minipage}{.35\textwidth}
      \begin{tcolorbox}[colframe=gray,colback=white,boxrule=1pt,arc=3.4pt,boxsep=0mm]
        \begin{minted}{csharp}
          class Reader<TC, TA> {
            Reader(T runReader) {
              this.runReader = runReader;
            }
          }
        \end{minted}
      \end{tcolorbox}
    \end{minipage}%
  }
  \vspace*{2.5cm}
  \begin{block}{\textbf{Exercise: }Reader Type Definition}
    \begin{align*}
      \mathsf{newtype}\ \ReaderH{a} &=
        \mathsf{Reader} \left\{ \fn{runReader} ::\ \uncover<2->{\mylib{c}\ \rightarrow\ \mylibo{a}} \right\}\\
    \end{align*}
  \end{block}
\end{frame}
\begin{frame}[fragile]{Reader Monad}{Type}
  Recall: $\mathsf{newtype}\ \Reader{a} = \mathsf{Reader} \left\{ \fn{runReader} ::\ c\ \rightarrow\ a \right\}$
  \begin{block}{\textbf{Exercise: }Reader Type Definition}
    \begin{align*}
      \return &:: a\ \rightarrow\ \Reader{a}\\
      \return\ a\ &= \uncover<2->{\mathsf{Reader}\ (\lambda\,\underline{\phantom{c}}\ \rightarrow\ a)}\\[1.5em]
      (\bind) &:: \Reader{a}\ \rightarrow\ (a\ \rightarrow\ \Reader{b})\ \rightarrow\ \Reader{b}\\
      x\ \bind\ fn &=\uncover<3->{\mathsf{Reader}\ (\lambda\,c\ \rightarrow
      \fn{runReader}\ (fn\ (\fn{runReader}\ x\ c))\ c)}\\
    \end{align*}
  \end{block}
\end{frame}
\begin{frame}[fragile]{Reader Monad}{Functions}
    \begin{block}{\textbf{Homework: }Reader Functions}
      \begin{align*}
        \fn{ask} &:: \Reader{a}\ \rightarrow\ a\\
        \fn{asks} &:: \Reader{a}\ \rightarrow\ (a\ \rightarrow\ \Reader{b})\ \rightarrow\ \Reader{b}\\
        \fn{local} &:: (c\ \rightarrow\ c\,')\ \rightarrow\ \mathbf{Reader}\,c\,'\,a\ \rightarrow\ \Reader{a}\\
      \end{align*}
    \end{block}
\end{frame}
\subsection{Example}
\begin{frame}
  \centering
  \begin{tcolorbox}[enhanced, size=minimal,auto outer arc,
    width=2.8cm,octogon arc, colback=red,colframe=white,colupper=white, fontupper=\fontsize{7mm}{7mm}\selectfont\bfseries\sffamily, halign=center,valign=center,
    square,arc is angular, borderline={1mm}{-3mm}{red} ]
    CODE
  \end{tcolorbox}
\end{frame}
%
  \section{State}
\subsection{Motivation}
\begin{frame}[fragile]{State Monad}{Motivation}
  \begin{columns}[t]
    \begin{column}{.55\textwidth}
      {\tiny\inputminted[escapeinside=\\`\\`,breaklines,breakafter=>]{haskell}{state-motivation.hs}}
    \end{column}
    \begin{column}{.4\textwidth}
      \begin{itemize}
        \item Notice how we must pass this \texttt{\mylib{gs*}} around
        \item Again, we must pass it through functions that may never use it
        \item And this time we get back a new version which we must remember to use so we do not lose state
      \end{itemize}
    \end{column}
  \end{columns}
\end{frame}
\subsection{Definition}
\begin{frame}[fragile]{State Monad}{Type}
    {\scriptsize\inputminted[escapeinside=||]{haskell}{state-types.hs}}
    \begin{block}{\textbf{Exercise: }Reader Type Definition}
      \begin{align*}
        \mathsf{newtype}\ \StateH{a} &=
          \mathsf{State} \left\{ \fn{runState} ::\ \uncover<2->{\mylib{s}\ \rightarrow\ (\mylib{s}, \mylibo{a})} \right\}\\
      \end{align*}
    \end{block}
\end{frame}
\begin{frame}[fragile]{State Monad}{Type}
    Recall: $\mathsf{newtype}\ \State{a} = \mathsf{State} \left\{ \fn{runState} ::\ s\ \rightarrow\ (s, a) \right\}$
    \begin{block}{\textbf{Exercise: }State Type Definition}
      \begin{align*}
        \return &:: a\ \rightarrow\ \State{a}\\
        \return\ a\ &= \uncover<2->{\mathsf{State}\ (\lambda\,s\ \rightarrow\ (s, a))}\\[1.5em]
        (\bind) &:: \State{a}\ \rightarrow\ (a\ \rightarrow\ \State{b})\ \rightarrow\ \State{b}\\
        x\ \bind\ fn &=\uncover<3->{\mathsf{State}\ (\lambda\,s\ \rightarrow\ 
          \begin{aligned}[t]
            \mathbf{let}&\ (s\,', a) = \fn{runState}\ x\ s\\
            \mathbf{in}&\ \fn{runState}\ (fn\ a)\ s\,')
          \end{aligned}}\\
      \end{align*}
    \end{block}
\end{frame}
\begin{frame}[fragile]{State Monad}{Functions}
    \begin{block}{\textbf{Homework: }State Functions}
      \begin{align*}
        \fn{get} &:: \State{s}\\
        \fn{gets} &:: (s\ \rightarrow\ a)\ \rightarrow\ \State{a}\\
        \fn{put} &:: s\ \rightarrow\ \State{\unit}\\
        \fn{modify} &:: (s\ \rightarrow\ s)\ \rightarrow\ \State{\unit}\\
      \end{align*}
    \end{block}
\end{frame}
\subsection{Example}
\begin{frame}
  \centering
  \begin{tcolorbox}[enhanced, size=minimal,auto outer arc,
    width=2.8cm,octogon arc, colback=red,colframe=white,colupper=white, fontupper=\fontsize{7mm}{7mm}\selectfont\bfseries\sffamily, halign=center,valign=center,
    square,arc is angular, borderline={1mm}{-3mm}{red} ]
    CODE
  \end{tcolorbox}
\end{frame}
%
  \section{Writer}
\subsection{Motivation}
\begin{frame}[fragile]{Writer Monad}{Motivation}
  {\tiny\inputminted[escapeinside=||,breakafter=\$]{haskell}{writer-motivation.hs}}
  \begin{itemize}
    \item Notice how we must add onto this list of strings
    \item Functions don't care about the logs from what proceeded them, so why should we have to know about them
  \end{itemize}
\end{frame}
\subsection{Definition}
\begin{frame}[fragile]{Writer Monad}{Type}
    {\scriptsize\inputminted[escapeinside=||]{haskell}{writer-types.hs}}
    \begin{block}{\textbf{Exercise: }Writer Type Definition}
      \begin{align*}
        \mathsf{writer}\ \WriterH{a} &= \mathsf{Writer} \left\{ \fn{runWriter} ::\ \uncover<2->{(\mylib{$[\ell]$}, \mylibo{a})} \right\}\\
      \end{align*}
    \end{block}
\end{frame}
\begin{frame}[fragile]{Writer Monad}{Type}
  Recall: $\mathsf{writer}\ \Writer{a} = \mathsf{Writer} \left\{ \fn{runWriter} ::\ ([\ell], a) \right\}$
  \begin{block}{\textbf{Exercise: }Writer Type Definition}
    \begin{align*}
      \return &:: a\ \rightarrow\ \Writer{a}\\
      \return\ a\ &= \uncover<2->{\mathsf{Writer}\ ([\,], a)}\\[1.5em]
      (\bind) &::
      \begin{aligned}[t]
        \Writer{a}\ \rightarrow\ &{\text{\small\color{gray}\carriagereturn}}\\(a\ \rightarrow\ &\Writer{b})\ \rightarrow\ \Writer{b}
      \end{aligned}\\
      x\ \bind\ fn &=\uncover<3->{\mathsf{Writer}\ (
          \begin{aligned}[t]
            \mathbf{let}&\ 
            \begin{aligned}[t]
              (l, a) &= \fn{runWriter}\ x\\
              (l\,', b) &= \fn{runWriter}\ (fn\ a)
            \end{aligned}\\
            \mathbf{in}&\ (l\!\mappend\!l\,', b))
          \end{aligned}}\\
    \end{align*}
  \end{block}
\end{frame}
\begin{frame}[fragile]{Writer Monad}{Monoid}
  Let: $\mathsf{writer}\ \WriterM{a} = \mathsf{Writer} \left\{ \fn{runWriter} ::\ (\ell, a) \right\}$
  \begin{block}{\textbf{Exercise: }Writer Type Definition}
    \begin{align*}
      \return &:: a\ \rightarrow\ \WriterM{a}\\
      \return\ a\ &= \mathsf{Writer}\ (\fn{mempty}, a)\\[1.5em]
      (\bind) &::
      \begin{aligned}[t]
        \WriterM{a}\ \rightarrow\ &{\text{\small\color{gray}\carriagereturn}}\\(a\ \rightarrow\ &\WriterM{b})\ \rightarrow\ \WriterM{b}
      \end{aligned}\\
      x\ \bind\ fn &=\mathsf{Writer}\ (
          \begin{aligned}[t]
            \mathbf{let}&\ 
            \begin{aligned}[t]
              (l, a) &= \fn{runWriter}\ x\\
              (l\,', b) &= \fn{runWriter}\ (fn\ a)
            \end{aligned}\\
            \mathbf{in}&\ (\fn{mappend}\ l\ l\,', b))
          \end{aligned}\\
    \end{align*}
  \end{block}
\end{frame}
\begin{frame}[fragile]{Writer Monad}{Functions}
    \begin{block}{\textbf{Homework: }Writer Functions}
      \begin{align*}
        \fn{tell} &:: \ell\ \rightarrow\ \WriterM{\unit}\\
        \fn{censor} &:: (\ell\ \rightarrow\ \ell)\ \rightarrow\ \WriterM{a}\ \rightarrow\ \WriterM{a}\\
        \fn{listen} &:: \WriterM{a}\ \rightarrow\ \WriterM{(a, \ell)}\\
      \end{align*}
    \end{block}
\end{frame}
\subsection{Example}
\begin{frame}
  \centering
  \begin{tcolorbox}[enhanced, size=minimal,auto outer arc,
    width=2.8cm,octogon arc, colback=red,colframe=white,colupper=white, fontupper=\fontsize{7mm}{7mm}\selectfont\bfseries\sffamily, halign=center,valign=center,
    square,arc is angular, borderline={1mm}{-3mm}{red} ]
    CODE
  \end{tcolorbox}
\end{frame}

%
  \section{It's Trevor Hunting Time}
\begin{frame}[fragile]
  \fontsize{30pt}{1em}\selectfont
  \vspace*{-260pt}
  \hspace*{-225pt}
  \begin{tikzpicture}[scale=1.3, every shadow/.style={opacity=1,fill=blue!10!black}]
    \foreach \l in {300.8,280.8,...,60.8} {%
      \path[circular glow={shadow scale=1.03}, shading=radial, inner color=yellow!80!white, outer color=red!50!black] (0, 0) circle (\l pt);
    }

    \begin{scope}
      \path[circular glow={shadow scale=1.03}, shading=radial, inner color=blue!25!black, outer color=darkblueOuter,clip] (0, 0) circle (60pt)
      node[circle,inner sep=60pt,fill overzoom image=wumpus] at (0pt, -35pt) {};
    \end{scope}

    \node (b) at (-3, -0.8) {};
    \node (e) at (10, -1.2) {};
    \draw[decoration={text along path, text color=white, text={That's all Folks!}}, decorate] (b) to[bend left=15] (e);
  \end{tikzpicture}
\end{frame}
%
\end{document}
