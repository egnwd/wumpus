\section{State}
\subsection{Motivation}
\begin{frame}[fragile]{State Monad}{Motivation}
  \begin{columns}[t]
    \begin{column}{.55\textwidth}
      {\tiny\inputminted[escapeinside=\\`\\`,breaklines,breakafter=>]{haskell}{state-motivation.hs}}
    \end{column}
    \begin{column}{.4\textwidth}
      \begin{itemize}
        \item Notice how we must pass this \texttt{\mylib{gs*}} around
        \item Again, we must pass it through functions that may never use it
        \item And this time we get back a new version which we must remember to use so we do not lose state
      \end{itemize}
    \end{column}
  \end{columns}
\end{frame}
\subsection{Definition}
\begin{frame}[fragile]{State Monad}{Type}
    {\scriptsize\inputminted[escapeinside=||]{haskell}{state-types.hs}}
    \begin{block}{\textbf{Exercise: }Reader Type Definition}
      \begin{align*}
        \mathsf{newtype}\ \StateH{a} &=
          \mathsf{State} \left\{ \fn{runState} ::\ \uncover<2->{\mylib{s}\ \rightarrow\ (\mylib{s}, \mylibo{a})} \right\}\\
      \end{align*}
    \end{block}
\end{frame}
\begin{frame}[fragile]{State Monad}{Type}
    Recall: $\mathsf{newtype}\ \State{a} = \mathsf{State} \left\{ \fn{runState} ::\ s\ \rightarrow\ (s, a) \right\}$
    \begin{block}{\textbf{Exercise: }State Type Definition}
      \begin{align*}
        \return &:: a\ \rightarrow\ \State{a}\\
        \return\ a\ &= \uncover<2->{\mathsf{State}\ (\lambda\,s\ \rightarrow\ (s, a))}\\[1.5em]
        (\bind) &:: \State{a}\ \rightarrow\ (a\ \rightarrow\ \State{b})\ \rightarrow\ \State{b}\\
        x\ \bind\ fn &=\uncover<3->{\mathsf{State}\ (\lambda\,s\ \rightarrow\ 
          \begin{aligned}[t]
            \mathbf{let}&\ (s\,', a) = \fn{runState}\ x\ s\\
            \mathbf{in}&\ \fn{runState}\ (fn\ a)\ s\,')
          \end{aligned}}\\
      \end{align*}
    \end{block}
\end{frame}
\begin{frame}[fragile]{State Monad}{Functions}
    \begin{block}{\textbf{Homework: }State Functions}
      \begin{align*}
        \fn{get} &:: \State{s}\\
        \fn{gets} &:: (s\ \rightarrow\ a)\ \rightarrow\ \State{a}\\
        \fn{put} &:: s\ \rightarrow\ \State{\unit}\\
        \fn{modify} &:: (s\ \rightarrow\ s)\ \rightarrow\ \State{\unit}\\
      \end{align*}
    \end{block}
\end{frame}
\subsection{Example}
\begin{frame}
  \centering
  \begin{tcolorbox}[enhanced, size=minimal,auto outer arc,
    width=2.8cm,octogon arc, colback=red,colframe=white,colupper=white, fontupper=\fontsize{7mm}{7mm}\selectfont\bfseries\sffamily, halign=center,valign=center,
    square,arc is angular, borderline={1mm}{-3mm}{red} ]
    CODE
  \end{tcolorbox}
\end{frame}
